\RequirePackage{plautopatch}
% \RequirePackage[l2tabu, orthodox]{nag}

\documentclass{ltjsarticle}
% \documentclass[platex,dvipdfmx]{jlreq}            % for platex
% \documentclass[uplatex,dvipdfmx]{jlreq}        % for uplatex
% \usepackage{graphicx}
% \usepackage{bxtexlogo}
% \usepackage{lmodern}
\usepackage{amsmath, amssymb, amsthm}
% \usepackage{mathtools}
% \usepackage{mathrsfs}
% \usepackage{bm}
% \usepackage{mdframed}
\usepackage{hyperref}
\usepackage{xcolor}
\usepackage{framed}
\usepackage{tikz}
\setlength{\parindent}{0pt}

\makeatletter
\renewenvironment{leftbar}{%
  \renewcommand\FrameCommand{\vrule width 1pt \hspace{10pt}}%
  \MakeFramed {\advance\hsize-\width \FrameRestore}}%
 {\endMakeFramed}
\makeatother

\newcommand{\barquo}[1]{\begin{leftbar} \noindent #1 \end{leftbar}} % 左線つき引用

\definecolor{lightgray}{gray}{0.9}

\title{PRML 1章練習問題解答}
\author{\href{https://github.com/Taka0007}{Taka007}}
\date{最終更新日:\today}
\begin{document}
\maketitle
\subsubsection*{1-1}



\subsubsection*{1-6}

$
E_{x,y} [ \{x-E[x] \}  \{y-E[y] \}]
\\
= E[ xy - x \cdot E[y]  -  y \cdot E[x]   +E[x]E[y]  ]
\\
 = E[xy] - E[x \cdot E[y]]  - E[y \cdot E[x]]  + E[x]E[y]
\\
E[x],E[y]は定数であり、外に出せるため \\
= E[xy] -E[y] E[x] - E[x] E[y] + E[x]E[y]
\\
= E[xy]  -  E[x]E[y]
$

$x,y$が独立であるとき、
$E[xy]= E[x]E[y]  $が成立する。

よって$E[xy]  -  E[x]E[y] = 0$となり、$cov[x,y]=E[xy]  -  E[x]E[y] = 0$



\subsubsection*{参考文献}

[1] Bishop, C. M. (2006). Pattern Recognition and Machine Learning. Springer.



\end{document}

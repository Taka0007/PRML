\RequirePackage{plautopatch}
% \RequirePackage[l2tabu, orthodox]{nag}

\documentclass{ltjsarticle}
% \documentclass[platex,dvipdfmx]{jlreq}            % for platex
% \documentclass[uplatex,dvipdfmx]{jlreq}        % for uplatex
% \usepackage{graphicx}
% \usepackage{bxtexlogo}
% \usepackage{lmodern}
\usepackage{amsmath, amssymb, amsthm}
% \usepackage{mathtools}
% \usepackage{mathrsfs}
% \usepackage{bm}
% \usepackage{mdframed}
\usepackage{hyperref}
\usepackage{xcolor}
\setlength{\parindent}{0pt}

\definecolor{lightgray}{gray}{0.9}

\title{PRML 2章練習問題解答}
\author{\href{https://github.com/Taka0007}{Taka007}}
\date{最終更新日:\today}
\begin{document}
\maketitle

\tableofcontents
\newpage

\section*{2-1}
\addcontentsline{toc}{section}{2-1}
[1]ベルヌーイ分布は正規化されている。
つまり下記が成立することを示せ。
$$
\sum_{n=0}^1 p(x| \mu ) = 1
$$

ベルヌーイ分布は
$
Ber(x| \mu) = \mu^x (1-\mu)^{1-x}
$
となる。

$
\sum_{n=0}^1 Ber(x| \mu)
= Ber(0| \mu) + Ber(1| \mu)
= \mu^0 (1-\mu)^{1-0} + \mu^1 (1-\mu)^{1-1}
= 1 \times (1-\mu) + \mu \times 1 \\
= 1-\mu + \mu
= \underline{1}
$
\\
上記より、
$$
\sum_{n=0}^1 Ber(x| \mu) = 1
$$
が成立するので、ベルヌーイ分布は正規化されている。
\\

[2] $E[x] = \mu $であることを示す。\\
$
E[x] = 0 \times Ber(0| \mu) + 1 \times Ber(1| \mu)
= 0 \times (1-\mu) + 1 \times \mu
= \underline{\mu}
$
\\

[3] $V[x] = \mu (1-\mu)$であることを示す。\\
$$
V[x] = E[x^2] - E[x]^2
$$
\\
ここで、
$
E[x^2] = 0^2 \times Ber(0| \mu) + 1^2 \times Ber(1| \mu)
= 0 \times (1-\mu) + 1 \times \mu
= \underline{\mu}
$

よって、
$
V[x] = \mu - \mu^2
= \underline{\mu (1-\mu)}
$



\section*{2-12}
\addcontentsline{toc}{section}{2-12}
一様分布は下記のように定義される。 \\
$$
U(x|a,b) = \frac{1}{b-a} \ \ \ \ (a \leq x \leq b)
$$






\section*{2-26}
\addcontentsline{toc}{section}{2-26}
$A,B,C,D$は行列とする。\\
$
(A + BCD)^{-1} 
=  A^{-1} -  A^{-1}B (C^{-1} + D A^{-1} B)^{-1} D A^{-1}
$
を示す。

右辺に
$
(A + BCD)
$
を右から掛ける。

$
(A^{-1} -  A^{-1}B (C^{-1} + D A^{-1} B)^{-1} D A^{-1}) (A + BCD)
= 
$












\section*{参考文献}
\addcontentsline{toc}{section}{参考文献}
[1] Bishop, C. M. (2006). Pattern Recognition and Machine Learning. Springer.



\end{document}
